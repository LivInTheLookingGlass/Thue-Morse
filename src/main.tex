\documentclass[conference]{IEEEtran}
\def\BibTeX{{\rm B\kern-.05em{\sc i\kern-.025em b}\kern-.08em
    T\kern-.1667em\lower.7ex\hbox{E}\kern-.125emX}}
\usepackage{amsmath,amssymb,amsfonts}
\usepackage{algorithmic}
\usepackage{graphicx}
\usepackage{textcomp}
\usepackage{xcolor}
\usepackage{xspace}
\usepackage[numbers,sort&compress]{natbib}
\usepackage{stmaryrd}
\usepackage{listings}
\usepackage{verbatim}
\usepackage{xcolor}  % for setting colors
\usepackage{url}
\usepackage{float}
\usepackage{multirow}
\usepackage{colortbl}
% \usepackage{showlabels}
\usepackage{amsthm}
\usepackage{tikz}

%counters
\newcounter{stddefctr}
\setcounter{stddefctr}{1}
\newcounter{extdefctr}
\setcounter{extdefctr}{1}

% Define Python style for listings
\lstdefinestyle{pythonstyle}{
    language=Python,
    basicstyle=\ttfamily\small,
    keywordstyle=\color{blue},
    commentstyle=\color{green},
    stringstyle=\color{red},
    frame=none,
    breaklines=true,
    morekeywords={yield}
}

% Edit the mod command to take less space
\let\oldmod\mod
\renewcommand{\mod}[1]{\hspace{-7pt} \oldmod \, #1}

% Define the note command
\newcommand{\note}[1]{\small\textcolor{purple}{#1}\normalsize}

% Define a concat command
\newcommand*\concat{\mathbin{\|}}

% Define a macro for the floor function
\newcommand{\floor}[1]{\left\lfloor #1 \right\rfloor}

% Define a macro for the ceiling function
\newcommand{\ceil}[1]{\left\lceil #1 \right\rceil}

% Define a macro for a tuple using angled brackets
\newcommand{\tuple}[1]{\langle #1 \rangle}

% Shorthands
\newcommand{\TMS}{Thue-Morse Sequence\xspace}
\newcommand{\ETMS}{Extended Thue-Morse Sequence\xspace}
\newcommand{\Integers}{\mathbb{Z}}
\newcommand{\Naturals}{\mathbb{N}}
\newcommand{\totaloriginals}{15}
\newcommand{\totalextensions}{9}
\newcommand{\TotalOriginals}{\totaloriginals\xspace}
\newcommand{\TotalExtensions}{\totalextensions\xspace}
\newcommand{\TotalDefs}{\the\numexpr \totaloriginals + \totalextensions \relax\xspace}


\begin{document}

\title{Extending The \TMS}

\author{\IEEEauthorblockN{Olivia Appleton}
\IEEEauthorblockA{\textit{TMW Center for Early Learning + Public Health}\\
\textit{University of Chicago}\\
Chicago, Illinois, United States\\
ORCID: 0009-0004-2296-7033}
\and
\IEEEauthorblockN{Dan Rowe}
\IEEEauthorblockA{\textit{Department of Math and Computer Science}\\
Northern Michigan University\\
Marquette, Michigan, United States\\
Email: darowe@nmu.edu}
}

\maketitle

\begin{abstract}
In this paper, we discuss various ways to extend the \TMS \cite{OEIS-TMS} when used as a fair-share sequence. Included are N definitions of the original sequence, M extensions to n players, and proofs of equality for all definitions. In the appendix are several complexity analyses for both space and time of each definition.
\end{abstract}

\begin{IEEEkeywords}
TBA
\end{IEEEkeywords}

\section{Introduction}

\note{Make sure to add that while this paper does not deal with negative or fractional bases, many of the definitions are trivially extendable to that domain, and at least one of them already has been in another paper.}

\section{The Original Sequence}

\subsection{Definition 1 - Parity of Hamming Weight}

\note{This definition appears in \cite{Spiegelhofer_2020, Allouche-Shallit_1999, OEIS-TMS}}

The Hamming Weight, as typically defined, is the digit sum of a binary number. In other words, it is a count of the high bits in a given number. A common way to generate the \TMS is to take the parity of the Hamming Weight for each natural number. We can use that as follows:

\begin{equation}
    \begin{aligned}
        p(0) &= 0 \\
        p(n) &= n + p\left(\floor{\dfrac{n}{2}}\right) \pmod{2}
    \end{aligned}
\end{equation}

Under this definition, you can construct the \TMS using the following, starting at $0$:

\begin{equation}
    T_{2,1}(n) = p(n)
\end{equation}

The subscript indicates that we are using $2$ players (writing in base $2$) and that we are using the first definition laid out in this paper. Note that when we extend to n players, the $T$ function will get a second parameter for the number of players, so it will look like $T_{n,d}(x, s)$, where $s$ is the size of the player pool, and therefore the base we use to define the sequence.

\subsection{Definition 2 - Invert and Extend}

\note{Appears in \cite{OEIS-TMS}}

This definition is more natural to think about as extending a tuple that contains the sequence. We will give a recurrence relation below, but to build an intuition we will work in this framework first.

Let $t(n)$ be the first $2^n$ elements of the \TMS. Given this, we can define:

\begin{equation}
    \text{inv}(\mathbf{x}) = \begin{cases}
        0, & \text{if } x_i = 1 \\
        1, & \text{if } x_i = 0
    \end{cases} \quad \text{for } \mathbf{x} = (x_0, x_1, \ldots, x_{n-1})
\end{equation}

\begin{equation}
    \begin{aligned}
        t(0) &= \tuple{0} \\
        t(n) &= t(n - 1) \concat \text{inv}(t(n - 1))
    \end{aligned}
\end{equation}

Given the above, we can define a recurrence relation that will give us individual elements. It will be less efficient to compute, but will allow proofs of equivalence to be easier.

\begin{equation}
    \begin{aligned}
        T_{2,2}(0) &= 0 \\
        T_{2,2}(n) &= T_{2,2}\left(n - 2^{\floor{\log_2(n)}}\right) \pm 1 \pmod{2}
    \end{aligned}
\end{equation}

\subsection{Definition 3 - Substitute and Flatten}

\note{This definition appears in \cite{Spiegelhofer_2020, Kolář-Nori_1991, OEIS-TMS}}

\[\bigparallel_{i=0}^{n-1} x_i\]

\subsection{Definition 4 - Recursion}

\note{This definition appears in \cite{Kolář-Nori_1991, OEIS-TMS}}

\begin{equation}
    \begin{aligned}
        T_{2,4}(0) &= 0 \\
        T_{2,4}(2n) &= T_{2,4}(n) \\
        T_{2,4}(2n+1) &= 1 - T_{2,4}(n) \pmod{2}
    \end{aligned}
\end{equation}

\subsection{Definition 5 - Highest Bit Difference}

\note{This definition appears in \cite{Arndt_2010}}

\note{The text below is from Wiki and needs to be entirely rewritten. I was able to derive the formula on my own from translating their code. This method leads to a fast method for computing the Thue–Morse sequence: start with t0 = 0, and then, for each n, find the highest-order bit in the binary representation of n that is different from the same bit in the representation of n - 1. If this bit is at an even index, tn differs from tn - 1, and otherwise it is the same as tn - 1.}

\begin{equation}
    \begin{aligned}
        T_{2,5}(0) &= 0 \\
        T_{2,5}(n) &= \begin{aligned}[c]
            &\floor{\log_2(n \oplus (n - 1))} \\
            &+ T_{2,5}(n - 1) \pm 1
        \end{aligned} \pmod{2}
    \end{aligned}
\end{equation}

\subsection{Definition 6 - Odious Number Derivation}

\note{Definition appears in \cite{OEIS-TMS}}

Another way to generate the \TMS is to take the sequence of Odious Numbers \cite{OEIS-Odious} mod $2$. Odious numbers are those with an odd number of $1$s in their binary representation. Note that the player numbers in this derivation are swapped, so when generating this for testing and extension, we add 1 to the result. Some simple generating code for this is as follows:

\begin{lstlisting}[style=pythonstyle]
from itertools import count

def seq_p2_d06():
    for i in count():
        if bin(i).count('1') % 2:
            yield (i + 1) % 2
\end{lstlisting}

\note{Aren't Odious Numbers exactly the numbers where the parity of the hamming weight is 1? So doesn't that mean that the Thue-Morse Sequence selects which numbers are Odious? From cursory testing, it seems to. There's something to be had there.}

\note{A possible way to extend this would be to reinterpret this as where the digit sum is not n-even}

\subsection{Summary}

\section{The Extensions}

\subsection{Definition 1 - Modular Digit Sums}

\note{Definition appears in \cite{Astudillo_2003, Dekking_2023}}

To extend definition 1 from $2$ to $n$ players, we must first map our concept of parity to base n. We can do this by taking the parity equation defined above and replacing the $2$s with $n$, for $n \in \Integers_{\ge 2}$.

\begin{equation}
    \begin{aligned}
        p_n(0) &= 0 \\
        p_n(x) &= x + p_n\left(\floor{\dfrac{x}{n}}\right) \pmod{n}
    \end{aligned}
\end{equation}

Under this definition, you can construct the \TMS using the following, starting at 0:

\begin{equation}
    T_{n,1}(x, s) = p_s(x)
\end{equation}

\subsubsection{Proof of Equivalence with Original Definition 1}

It is clear from visual inspection that $p_2$ is identical to our original definition of $p$.

\begin{equation}
    \begin{aligned}
        p_2(x) &= p(x) \\
        x + p_2\left(\floor{\dfrac{x}{2}}\right) &= x + p\left(\floor{\dfrac{x}{2}}\right) \\
        x + \floor{\dfrac{x}{2}} + p_2\left(\floor{\dfrac{x}{2^2}}\right) &= x + \floor{\dfrac{x}{2}} + p\left(\floor{\dfrac{x}{2^2}}\right) \\
        x + \floor{\dfrac{x}{2}} + \floor{\dfrac{x}{2^2}} + \ldots &= x + \floor{\dfrac{x}{2}} + \floor{\dfrac{x}{2^2}} + \ldots
    \end{aligned}
\end{equation}

\subsection{Definition 2 - Increment and Extend}

In the original version of this definition, we inverted the elements. In base $2$, this is the same thing as adding $1$ (mod $2$). Given that, let $t(x, n)$ be the first $n^x$ elements of the \ETMS, for $n \in \Integers_{\ge 2}$.

\begin{equation}
    \text{inc}(\mathbf{x}, n) = \begin{aligned}[c]
            &x_i + 1 \pmod{n} \\
            &\text{for } \mathbf{x} = (x_0, x_1, \ldots, x_{n-1})
    \end{aligned}
\end{equation}

\begin{equation}
    \begin{aligned}
        t(0, n) &= \tuple{0} \\
        t(1, n) &= \tuple{0, 1, \ldots, n - 1} \\
        t(x, n) &= t(x - 1, n) \cdot \text{inc}(t(x - 1, n), n)
    \end{aligned}
\end{equation}

Given the above, we can define a recurrence relation that will give us individual elements. It will be less efficient to compute, but will allow proofs of equivalence to be easier.

\begin{equation}
    \begin{aligned}
        T_{n,2}(0, s) &= 0 \\
        T_{n,2}(x, s) &= T_{n,2}\left(x - s^{\floor{\log_s(x)}}, s\right) + 1 \pmod{s}
    \end{aligned}
\end{equation}

\subsubsection{Proof of Equivalence with Original Definition 2}

\subsection{Definition 3 - Substitute and Flatten}

\subsubsection{Proof of Equivalence with Original Definition 3}

\subsection{Definition 4 - Recursion}

\subsubsection{Proof of Equivalence with Original Definition 4}

\subsection{Definition 5 - Highest Digit Difference}

\note{This is speculative, and testing needs to be done}

\subsubsection{Proof of Equivalence with Original Definition 5}

\subsection{Definition 6 - Odious Extension}

\note{This is speculative, and testing needs to be done}

\subsubsection{Proof of Equivalence with Original Definition 6}

\subsection{Summary}

\section{Proving Equivalence Between Extensions}

\subsection{Correlating Definition 1 and Definition 2}

\subsection{Correlating Definition 1 and Definition 3}

\subsection{Correlating Definition 1 and Definition 4}

\subsection{Summary}

\section{Proving Persistence of Original Properties}

\section{Acknowledgment}

The preferred spelling of the word ``acknowledgment'' in America is without 
an ``e'' after the ``g''. Avoid the stilted expression ``one of us (R. B. 
G.) thanks $\ldots$''. Instead, try ``R. B. G. thanks$\ldots$''. Put sponsor 
acknowledgments in the unnumbered footnote on the first page.

\section{Appendix}

\subsection{Complexity of Original Definition 1}

\subsubsection{Time Complexity}

\subsubsection{Space Complexity}

\subsection{Complexity of Original Definition 2}

\subsubsection{Time Complexity}

\subsubsection{Space Complexity}

\subsection{Complexity of Original Definition 3}

\subsubsection{Time Complexity}

\subsubsection{Space Complexity}

\subsection{Complexity of Original Definition 4}

\subsubsection{Time Complexity}

\subsubsection{Space Complexity}

\subsection{Complexity of Original Definition 5}

\subsubsection{Time Complexity}

\subsubsection{Space Complexity}

\subsection{Complexity of Original Definition 6}

\subsubsection{Time Complexity}

\subsubsection{Space Complexity}

\subsection{Complexity of Extension Definition 1}

\subsubsection{Time Complexity}

\subsubsection{Space Complexity}

\subsection{Complexity of Extension Definition 2}

\subsubsection{Time Complexity}

\subsubsection{Space Complexity}

\subsection{Complexity of Extension Definition 3}

\subsubsection{Time Complexity}

\subsubsection{Space Complexity}

\subsection{Complexity of Extension Definition 4}

\subsubsection{Time Complexity}

\subsubsection{Space Complexity}

\subsection{Complexity of Extension Definition 5}

\subsubsection{Time Complexity}

\subsubsection{Space Complexity}

\subsection{Complexity of Extension Definition 6}

\subsubsection{Time Complexity}

\subsubsection{Space Complexity}

\bibliographystyle{unsrt}
\bibliography{references}     % without the .bib extension

\end{document}
