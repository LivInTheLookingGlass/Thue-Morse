\documentclass[conference]{IEEEtran}
\IEEEoverridecommandlockouts
% The preceding line is only needed to identify funding in the first footnote. If that is unneeded, please comment it out.
\usepackage{cite}
\usepackage{amsmath,amssymb,amsfonts}
\usepackage{algorithmic}
\usepackage{graphicx}
\usepackage{textcomp}
\usepackage{xcolor}
\def\BibTeX{{\rm B\kern-.05em{\sc i\kern-.025em b}\kern-.08em
    T\kern-.1667em\lower.7ex\hbox{E}\kern-.125emX}}
\def\BibTeX{{\rm B\kern-.05em{\sc i\kern-.025em b}\kern-.08em
    T\kern-.1667em\lower.7ex\hbox{E}\kern-.125emX}}
\usepackage{amsmath,amssymb,amsfonts}
\usepackage{algorithmic}
\usepackage{graphicx}
\usepackage{textcomp}
\usepackage{xcolor}
\usepackage{xspace}
\usepackage[numbers,sort&compress]{natbib}
\usepackage{stmaryrd}
\usepackage{listings}
\usepackage{verbatim}
\usepackage{xcolor}  % for setting colors
\usepackage{url}
\usepackage{float}
\usepackage{multirow}
\usepackage{colortbl}
% \usepackage{showlabels}
\usepackage{amsthm}
\usepackage{tikz}

%counters
\newcounter{stddefctr}
\setcounter{stddefctr}{1}
\newcounter{extdefctr}
\setcounter{extdefctr}{1}

% Define Python style for listings
\lstdefinestyle{pythonstyle}{
    language=Python,
    basicstyle=\ttfamily\small,
    keywordstyle=\color{blue},
    commentstyle=\color{green},
    stringstyle=\color{red},
    frame=none,
    breaklines=true,
    morekeywords={yield}
}

% Edit the mod command to take less space
\let\oldmod\mod
\renewcommand{\mod}[1]{\hspace{-7pt} \oldmod \, #1}

% Define the note command
\newcommand{\note}[1]{\small\textcolor{purple}{#1}\normalsize}

% Define a concat command
\newcommand*\concat{\mathbin{\|}}

% Define a macro for the floor function
\newcommand{\floor}[1]{\left\lfloor #1 \right\rfloor}

% Define a macro for the ceiling function
\newcommand{\ceil}[1]{\left\lceil #1 \right\rceil}

% Define a macro for a tuple using angled brackets
\newcommand{\tuple}[1]{\langle #1 \rangle}

% Shorthands
\newcommand{\TMS}{Thue-Morse Sequence\xspace}
\newcommand{\ETMS}{Extended Thue-Morse Sequence\xspace}
\newcommand{\Integers}{\mathbb{Z}}
\newcommand{\Naturals}{\mathbb{N}}
\newcommand{\totaloriginals}{15}
\newcommand{\totalextensions}{9}
\newcommand{\TotalOriginals}{\totaloriginals\xspace}
\newcommand{\TotalExtensions}{\totalextensions\xspace}
\newcommand{\TotalDefs}{\the\numexpr \totaloriginals + \totalextensions \relax\xspace}



\begin{document}

\title{Extending The Thue-Morse Sequence}

\author{\IEEEauthorblockN{Olivia Appleton}
\IEEEauthorblockA{\textit{dept. name of organization (of Aff.)} \\
\textit{name of organization (of Aff.)}\\
Chicago, Illinois, United States \\
0009-0004-2296-7033}
\and
\IEEEauthorblockN{Dan Rowe?}
\IEEEauthorblockA{\textit{Department of Math and Computer Science} \\
\textit{Northern Michigan University}\\
Marquette, Michigan, United States \\
darowe@nmu.edu}
}

\maketitle

\begin{abstract}
In this paper, we discuss various ways to extend the Thue-Morse sequence when used as a fair-share sequence. Included are N definitions of the original sequence, M extensions to $n$ players, and proofs of equality for all definitions. In the appendix are several complexity analyses for both space and time of each definition.
\end{abstract}

\begin{IEEEkeywords}
component, formatting, style, styling, insert
\end{IEEEkeywords}

\section{Introduction}

\section{The Original Sequence}

\subsection{Definition 1 - Parity}

First, let us make sure we have a common definition of parity. For the purpose of this paper, this signifies whether there is an even or odd number of high bits in the binary representation of a number. It does not signify that the number as a whole is even or odd.

\begin{equation}
    \begin{aligned}
        p(0) &= 0 \\
        p(n) &= n + p\left(\floor{\dfrac{n}{2}}\right) \pmod{2}
    \end{aligned}
\end{equation}

Under this definition, you can construct the Thue-Morse Sequence using the following, starting at $0$:

\begin{equation}
    T_{2,1}(n) = p(n)
\end{equation}

The subscript indicates that we are using $2$ players (writing in base $2$) and that we are using the first definition laid out in this paper. Note that when we extend to n players, the $T$ function will get a second parameter for the number of players, so it will look like $T_{n,d}(x, s)$, where $s$ is the size of the player pool.

\subsection{Definition 2 - Invert and Extend}

This definition is more natural to think about as extending a tuple that contains the sequence. We will give a recurrence relation below, but to build an intuition we will work in this framework first.

Let $t(n)$ be the first $2^n$ elements of the Thue-Morse Sequence. Given this, we can define:

\begin{equation}
    \text{inv}(\mathbf{x}) = \begin{cases}
        0, & \text{if } x_i = 1 \\
        1, & \text{if } x_i = 0
    \end{cases} \quad \text{for } \mathbf{x} = (x_0, x_1, \ldots, x_{n-1})
\end{equation}

\begin{equation}
    \begin{aligned}
        t(0) &= \tuple{0} \\
        t(n) &= t(n - 1) \cdot \text{inv}(t(n - 1))
    \end{aligned}
\end{equation}

Given the above, we can define a recurrence relation that will give us individual elements. It will be less efficient to compute, but will allow proofs of equivalence to be easier.

\begin{equation}
    \begin{aligned}
        T_{2,3}(0) &= 0 \\
        T_{2,3}(n) &= T_{2,3}\left(n - 2^{\floor{\log_2(n)}}\right) + 1 \pmod{2}
    \end{aligned}
\end{equation}

\subsection{Definition 3 - Substitute and Flatten}

\subsection{Definition 4 - }

\subsection{Definition 5 - }

\subsection{Summary}

\section{The Extensions}

\subsection{Definition 1 - Modular Digit Sums}

To extend definition 1 from $2$ to $n$ players, we must first map our concept of parity to base n. We can do this by taking the parity equation defined above and replacing the $2$s with $n$, as follows:

\begin{equation}
    \begin{aligned}
        p_n(0) &= 0 \\
        p_n(x) &= x + p_n\left(\floor{\dfrac{x}{n}}\right) \pmod{n}
    \end{aligned}
\end{equation}

Under this definition, you can construct the Thue-Morse Sequence using the following, starting at 0:

\begin{equation}
    T_{n,1}(x, s) = p_s(x)
\end{equation}

\subsubsection{Proof of Equivalence with Orig. Def. 1}

\subsection{Definition 2}

In the original version of this definition, we inverted the elements. In base $2$, this is the same thing as adding $1$ (mod $2$). Given that, let $t(x, n)$ be the first $n^x$ elements of the Extended Thue-Morse Sequence.

\begin{equation}
    \begin{aligned}
        \text{inc}(\mathbf{x}, n) &= x_i + 1 \pmod{n} \\
        &\text{for } \mathbf{x} = (x_0, x_1, \ldots, x_{n-1})
    \end{aligned}
\end{equation}

\begin{equation}
    \begin{aligned}
        t(0, n) &= \tuple{0} \\
        t(1, n) &= \tuple{0, 1, \ldots, n - 1} \\
        t(x, n) &= t(x - 1, n) \cdot \text{inc}(t(x - 1, n), n)
    \end{aligned}
\end{equation}

Given the above, we can define a recurrence relation that will give us individual elements. It will be less efficient to compute, but will allow proofs of equivalence to be easier.

\begin{equation}
    \begin{aligned}
        T_{n,3}(0, s) &= 0 \\
        T_{n,3}(x, s) &= T_{2,3}\left(x - s^{\floor{\log_s(x)}}\right) + 1 \pmod{s}
    \end{aligned}
\end{equation}

\subsubsection{Proof of Equivalence with Orig. Def. 2}

\subsection{Definition 3}

\subsubsection{Proof of Equivalence with Orig. Def. 3}

\subsection{Definition 4}

\subsubsection{Proof of Equivalence with Orig. Def. 4}

\subsection{Summary}

\section{Proving Equivalence Between Extensions}

\subsection{Correlating Def. 1 and Def. 2}

\subsection{Correlating Def. 1 and Def. 3}

\subsection{Correlating Def. 1 and Def. 4}

\subsection{Summary}

\section*{Acknowledgment}

The preferred spelling of the word ``acknowledgment'' in America is without 
an ``e'' after the ``g''. Avoid the stilted expression ``one of us (R. B. 
G.) thanks $\ldots$''. Instead, try ``R. B. G. thanks$\ldots$''. Put sponsor 
acknowledgments in the unnumbered footnote on the first page.

\section*{References}

Please number citations consecutively within brackets \cite{b1}. The 
sentence punctuation follows the bracket \cite{b2}. Refer simply to the reference 
number, as in \cite{b3}---do not use ``Ref. \cite{b3}'' or ``reference \cite{b3}'' except at 
the beginning of a sentence: ``Reference \cite{b3} was the first $\ldots$''

Number footnotes separately in superscripts. Place the actual footnote at 
the bottom of the column in which it was cited. Do not put footnotes in the 
abstract or reference list. Use letters for table footnotes.

Unless there are six authors or more give all authors' names; do not use 
``et al.''. Papers that have not been published, even if they have been 
submitted for publication, should be cited as ``unpublished'' \cite{b4}. Papers 
that have been accepted for publication should be cited as ``in press'' \cite{b5}. 
Capitalize only the first word in a paper title, except for proper nouns and 
element symbols.

For papers published in translation journals, please give the English 
citation first, followed by the original foreign-language citation \cite{b6}.

\section{Appendix}

\subsection{Complexity of Extension 1}

\subsubsection{Time Complexity}

\subsubsection{Space Complexity}

\subsection{Complexity of Extension 2}

\subsubsection{Time Complexity}

\subsubsection{Space Complexity}
\begin{thebibliography}{00}
\bibitem{b1} G. Eason, B. Noble, and I. N. Sneddon, ``On certain integrals of Lipschitz-Hankel type involving products of Bessel functions,'' Phil. Trans. Roy. Soc. London, vol. A247, pp. 529--551, April 1955.
\bibitem{b2} J. Clerk Maxwell, A Treatise on Electricity and Magnetism, 3rd ed., vol. 2. Oxford: Clarendon, 1892, pp.68--73.
\bibitem{b3} I. S. Jacobs and C. P. Bean, ``Fine particles, thin films and exchange anisotropy,'' in Magnetism, vol. III, G. T. Rado and H. Suhl, Eds. New York: Academic, 1963, pp. 271--350.
\bibitem{b4} K. Elissa, ``Title of paper if known,'' unpublished.
\bibitem{b5} R. Nicole, ``Title of paper with only first word capitalized,'' J. Name Stand. Abbrev., in press.
\bibitem{b6} Y. Yorozu, M. Hirano, K. Oka, and Y. Tagawa, ``Electron spectroscopy studies on magneto-optical media and plastic substrate interface,'' IEEE Transl. J. Magn. Japan, vol. 2, pp. 740--741, August 1987 [Digests 9th Annual Conf. Magnetics Japan, p. 301, 1982].
\bibitem{b7} M. Young, The Technical Writer's Handbook. Mill Valley, CA: University Science, 1989.
\end{thebibliography}
\vspace{12pt}
\color{red}
IEEE conference templates contain guidance text for composing and formatting conference papers. Please ensure that all template text is removed from your conference paper prior to submission to the conference. Failure to remove the template text from your paper may result in your paper not being published.

\end{document}
