\documentclass[conference]{IEEEtran}
\def\BibTeX{{\rm B\kern-.05em{\sc i\kern-.025em b}\kern-.08em
    T\kern-.1667em\lower.7ex\hbox{E}\kern-.125emX}}
\usepackage{amsmath,amssymb,amsfonts}
\usepackage{algorithmic}
\usepackage{graphicx}
\usepackage{textcomp}
\usepackage{xcolor}
\usepackage{xspace}
\usepackage[numbers,sort&compress]{natbib}
\usepackage{stmaryrd}
\usepackage{listings}
\usepackage{verbatim}
\usepackage{xcolor}  % for setting colors
\usepackage{url}
\usepackage{float}
\usepackage{multirow}
\usepackage{colortbl}
% \usepackage{showlabels}
\usepackage{amsthm}
\usepackage{tikz}

%counters
\newcounter{stddefctr}
\setcounter{stddefctr}{1}
\newcounter{extdefctr}
\setcounter{extdefctr}{1}
\newcounter{rowcount}
\setcounter{rowcount}{1}

% Define Python style for listings
\lstdefinestyle{pythonstyle}{
    language=Python,
    basicstyle=\ttfamily\small,
    keywordstyle=\color{blue},
    commentstyle=\color{green},
    stringstyle=\color{red},
    frame=none,
    breaklines=true,
    morekeywords={yield}
}

% Edit the mod command to take less space
\let\oldmod\mod
\renewcommand{\mod}[1]{\hspace{-7pt} \oldmod \, #1}

% Define the note command
\newcommand{\note}[1]{\small\textcolor{violet}{#1}\normalsize}

% Define a concat command
\newcommand*\concat{\mathbin{\|}}

% Define a macro for the floor function
\newcommand{\floor}[1]{\left\lfloor #1 \right\rfloor}

% Define a macro for the ceiling function
\newcommand{\ceil}[1]{\left\lceil #1 \right\rceil}

% Define a macro for a tuple using angled brackets
\newcommand{\tuple}[1]{\langle #1 \rangle}

% Shorthands
\newcommand{\TMS}{Thue-Morse Sequence\xspace}
\newcommand{\ETMS}{Extended Thue-Morse Sequence\xspace}
\newcommand{\Integers}{\mathbb{Z}}
\newcommand{\Naturals}{\mathbb{N}}
\newcommand{\totaloriginals}{20}
\newcommand{\totalextensions}{9}
\newcommand{\TotalOriginals}{\totaloriginals\xspace}
\newcommand{\TotalExtensions}{\totalextensions\xspace}
\newcommand{\TotalDefs}{\the\numexpr \totaloriginals + \totalextensions \relax\xspace}
\newcommand{\rc}{\!\!\therowcount\stepcounter{rowcount}\!\!}
\newcommand{\Lm}{\!\!L\!\!}
\newcommand{\Nm}{\!\!N\!\!}
\newcommand{\Om}{\!\!O\!\!}
\newcommand{\Sm}{\!\!S\!\!}
\newcommand{\Xm}{\!\!X\!\!}


\begin{document}

\title{Extending The \TMS}

\author{\IEEEauthorblockN{Olivia Appleton}
\IEEEauthorblockA{\textit{dept. name of organization (of Aff.)} \\
\textit{name of organization (of Aff.)}\\
Chicago, Illinois, United States \\
0009-0004-2296-7033}
\and
\IEEEauthorblockN{Dan Rowe?}
\IEEEauthorblockA{\textit{Department of Math and Computer Science} \\
\textit{Northern Michigan University}\\
Marquette, Michigan, United States \\
darowe@nmu.edu}
}

\maketitle

\begin{abstract}
In this paper, we discuss various ways to extend the \TMS when used as a fair-share sequence. Included are N definitions of the original sequence, M extensions to $n$ players, and proofs of equality for all definitions. In the appendix are several complexity analyses for both space and time of each definition.
\end{abstract}

\begin{IEEEkeywords}
component, formatting, style, styling, insert
\end{IEEEkeywords}

\section{Introduction}

\section{The Original Sequence}

\subsection{Definition 1 - Parity}

This definition appears in \cite{Spiegelhofer_2020, Allouche-Shallit_1999}

First, let us make sure we have a common definition of parity. For the purpose of this paper, this signifies whether there is an even or odd number of high bits in the binary representation of a number. It does not signify that the number as a whole is even or odd.

\begin{equation}
    \begin{aligned}
        p(0) &= 0 \\
        p(n) &= n + p\left(\floor{\dfrac{n}{2}}\right) \pmod{2}
    \end{aligned}
\end{equation}

Under this definition, you can construct the \TMS using the following, starting at $0$:

\begin{equation}
    T_{2,1}(n) = p(n)
\end{equation}

The subscript indicates that we are using $2$ players (writing in base $2$) and that we are using the first definition laid out in this paper. Note that when we extend to n players, the $T$ function will get a second parameter for the number of players, so it will look like $T_{n,d}(x, s)$, where $s$ is the size of the player pool.

\subsection{Definition 2 - Invert and Extend}

This definition is more natural to think about as extending a tuple that contains the sequence. We will give a recurrence relation below, but to build an intuition we will work in this framework first.

Let $t(n)$ be the first $2^n$ elements of the \TMS. Given this, we can define:

\begin{equation}
    \text{inv}(\mathbf{x}) = \begin{cases}
        0, & \text{if } x_i = 1 \\
        1, & \text{if } x_i = 0
    \end{cases} \quad \text{for } \mathbf{x} = (x_0, x_1, \ldots, x_{n-1})
\end{equation}

\begin{equation}
    \begin{aligned}
        t(0) &= \tuple{0} \\
        t(n) &= t(n - 1) \cdot \text{inv}(t(n - 1))
    \end{aligned}
\end{equation}

Given the above, we can define a recurrence relation that will give us individual elements. It will be less efficient to compute, but will allow proofs of equivalence to be easier.

\begin{equation}
    \begin{aligned}
        T_{2,2}(0) &= 0 \\
        T_{2,2}(n) &= T_{2,2}\left(n - 2^{\floor{\log_2(n)}}\right) + 1 \pmod{2}
    \end{aligned}
\end{equation}

\subsection{Definition 3 - Substitute and Flatten}

This definition appears in \cite{Spiegelhofer_2020, Kolář-Nori_1991}

\subsection{Definition 4 - Recursion}

This definition appears in \cite{Kolář-Nori_1991}

\begin{equation}
    \begin{aligned}
        T_{2,4}(0) &= 0 \\
        T_{2,4}(2n) &= T_{2,4}(n) \\
        T_{2,4}(2n+1) &= 1 - T_{2,4}(n) \pmod{2}
    \end{aligned}
\end{equation}

\subsection{Definition 5 - }

\subsection{Summary}

\section{The Extensions}

\subsection{Definition 1 - Modular Digit Sums}

To extend definition 1 from $2$ to $n$ players, we must first map our concept of parity to base n. We can do this by taking the parity equation defined above and replacing the $2$s with $n$, as follows:

\begin{equation}
    \begin{aligned}
        p_n(0) &= 0 \\
        p_n(x) &= x + p_n\left(\floor{\dfrac{x}{n}}\right) \pmod{n}
    \end{aligned}
\end{equation}

Under this definition, you can construct the \TMS using the following, starting at 0:

\begin{equation}
    T_{n,1}(x, s) = p_s(x)
\end{equation}

\subsubsection{Proof of Equivalence with Original Definition 1}

\subsection{Definition 2 - Increment and Extend}

In the original version of this definition, we inverted the elements. In base $2$, this is the same thing as adding $1$ (mod $2$). Given that, let $t(x, n)$ be the first $n^x$ elements of the \ETMS.

\begin{equation}
    \begin{aligned}
        \text{inc}(\mathbf{x}, n) &= x_i + 1 \pmod{n} \\
        &\text{for } \mathbf{x} = (x_0, x_1, \ldots, x_{n-1})
    \end{aligned}
\end{equation}

\begin{equation}
    \begin{aligned}
        t(0, n) &= \tuple{0} \\
        t(1, n) &= \tuple{0, 1, \ldots, n - 1} \\
        t(x, n) &= t(x - 1, n) \cdot \text{inc}(t(x - 1, n), n)
    \end{aligned}
\end{equation}

Given the above, we can define a recurrence relation that will give us individual elements. It will be less efficient to compute, but will allow proofs of equivalence to be easier.

\begin{equation}
    \begin{aligned}
        T_{n,2}(0, s) &= 0 \\
        T_{n,2}(x, s) &= T_{n,2}\left(x - s^{\floor{\log_s(x)}}\right) + 1 \pmod{s}
    \end{aligned}
\end{equation}

\subsubsection{Proof of Equivalence with Original Definition 2}

\subsection{Definition 3 - Substitute and Flatten}

\subsubsection{Proof of Equivalence with Original Definition 3}

\subsection{Definition 4 - }

\subsubsection{Proof of Equivalence with Original Definition 4}

\subsection{Summary}

\section{Proving Equivalence Between Extensions}

\subsection{Correlating Definition 1 and Definition 2}

\subsection{Correlating Definition 1 and Definition 3}

\subsection{Correlating Definition 1 and Definition 4}

\subsection{Summary}

\section{Proving Persistence of Original Properties}

\section{Acknowledgment}

The preferred spelling of the word ``acknowledgment'' in America is without 
an ``e'' after the ``g''. Avoid the stilted expression ``one of us (R. B. 
G.) thanks $\ldots$''. Instead, try ``R. B. G. thanks$\ldots$''. Put sponsor 
acknowledgments in the unnumbered footnote on the first page.

\section{Appendix}

\subsection{Complexity of Original Definition 1}

\subsubsection{Time Complexity}

\subsubsection{Space Complexity}

\subsection{Complexity of Original Definition 2}

\subsubsection{Time Complexity}

\subsubsection{Space Complexity}

\subsection{Complexity of Original Definition 3}

\subsubsection{Time Complexity}

\subsubsection{Space Complexity}

\subsection{Complexity of Original Definition 4}

\subsubsection{Time Complexity}

\subsubsection{Space Complexity}

\subsection{Complexity of Extension Definition 1}

\subsubsection{Time Complexity}

\subsubsection{Space Complexity}

\subsection{Complexity of Extension Definition 2}

\subsubsection{Time Complexity}

\subsubsection{Space Complexity}

\subsection{Complexity of Extension Definition 3}

\subsubsection{Time Complexity}

\subsubsection{Space Complexity}

\subsection{Complexity of Extension Definition 4}

\subsubsection{Time Complexity}

\subsubsection{Space Complexity}

\bibliographystyle{plainnat} % choose an appropriate style
\bibliography{references}     % without the .bib extension

\end{document}
